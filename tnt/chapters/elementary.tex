\chapter{Elementary Methods}

\section{Arithmetic Functions}

\begin{definition}
  A function is said to be arithmetic, if $f:\N\rightarrow\C$.
  \begin{itemize}
    \item $f$ is said to be additive if $f(mn)=f(m)+f(n)$ $\forall m,n$ such that $(m,n)=1$.
    \item $f$ is said to be multiplicative if $f(mn)=f(m)f(n)$ $\forall m,n$ such that $(m,n)=1$.
    \item $f$ is said to be completely additive or multiplicative if additive or multiplicative
      property holds for all $m,n\in\N$
  \end{itemize}
\end{definition}

\begin{examples} Some arithmetic functions:
  \begin{enumerate}
    \item $\omega:\N\rightarrow\C$, $\omega(n)=\#\text{distincty prime factors of }n$. \\
      Additive.
    \item $\Omega:\N\rightarrow\C$, $\Omega(n)=\#\text{prime factors of }n\text{ with multiplicity.}$ \\
      Completely additive.
    \item $\log:\N\rightarrow\C$ \\
      Completely additive.
    \item $\mu:\N\rightarrow\C$ - Mobius function
      \[\mu(n)=\begin{cases}
        (-1)^{\omega(n)} & n\text{ is squarefree} \\
        0 & \text{otherwise}
      \end{cases}\]
      Multiplicative.
    \item $\Lambda:\N\rightarrow\C$ - von Mongoldt function
      \[\Lambda(n)=\begin{cases}
        \log(p) & \text{if }n=p^{\alpha}\text{ for some prime p} \\
        0 & \text{otherwise}
      \end{cases}\]
      Neither multiplicative nor additive.
    \item $\lambda(n):\N\rightarrow\C$ - Lioville's function
      \[ \lambda(n)=\begin{cases}
        1 & n=1 \\
        (-1)^{\alpha_1+\alpha_2+\cdots+\alpha_k} & \alpha_i\text{ such that }n=\prod_{i\leq k}p_i^{\alpha_i}
      \end{cases}\]
      Completely multiplicative.
    \item $\varphi:\N\rightarrow\C$ - Euler's totient function
      \begin{eqnarray*}
        \varphi(n) &=& |\left\{ 1\leq k\leq n:(n,k)=1\right\}| \\
             &=& n\prod_{p|n}\left( 1- \frac{1}{p}\right) \text{, and }\varphi(1)=1.
      \end{eqnarray*}
      $\varphi(mn)=\varphi(m)\varphi(n)\frac{(m,n)}{\varphi(m,n)}$
    \item $f:\N\rightarrow\C$, $f(n)=n^{-s}$, $s\in\C$. \\
      Completely multiplicative.
  \end{enumerate}
\end{examples}

\begin{theorem}
  If $n\geq 1$, $\displaystyle{\sum_{d|n}} \mu(d)=I(n):=\left[\frac{1}{n}\right]$.
\end{theorem}

\begin{proof}
  When $n=1$, the summation is equal to $1=I(1)$.
  Let $n=\displaystyle{\prod_{i=1} ^m}p_i^{\alpha_i}$ for some $m\in\N$, and let
  $N=\displaystyle{\prod_{i=1} ^m}p_i$. Now,
  \begin{eqnarray*}
    \sum_{d|n}\mu(d) = \sum_{d|N}\mu(d) &=& 1+\sum_{1\leq i\leq m}\mu(p_i)+\sum_{1\leq i,j\leq m}\mu(p_ip_j)+\cdots+\mu(N) \\
                              &=& 1-\begin{pmatrix}m\\1\end{pmatrix}+\begin{pmatrix}m\\2\end{pmatrix}-\cdots+(-1)^m \\
                              &=& (1+(-1))^m = 0 = I(n)
  \end{eqnarray*}
\end{proof}

\begin{definition}[Number Field]
  A finite field extension of $Q$ is known as a number field.
\end{definition}

\begin{definition}
  An integer $\alpha$ is said to be algebraic if $f(\alpha)=0$ for some monic irreducible
  $f\in\Z[x]$.
\end{definition}

\begin{definition}
  Given a number field $K$, the set
  \[\theta_k:=\{\alpha\in K:\alpha\text{ is an algebraic integer}\}\]
  is known as its ring of integers.
\end{definition}

\begin{theorem}
  Given $\zeta(s)>0\forall s>1$, there are inifinetely many primes.
\end{theorem}

\begin{proof}
  By Euler product formula,
  \begin{eqnarray*}
    \log\zeta(s) &=& \sum_p \log\left(1-\frac{1}{p^s}\right) = \sum_p\sum_{n=1}  ^{\infty}\frac{1}{np^{ns}} \\
             &=& \sum_p\frac{1}{p^s}+\sum_p\sum_{n\geq2}\frac{1}{np^{ns}}
  \end{eqnarray*}
  If $s>1$,
  \begin{eqnarray*}
    \sum_p\sum_{n\geq2}\frac{1}{np^{ns}} &\leq& \sum_p\sum_{n\geq2}\frac{1}{p^n} \\
                               &=& \sum_p \frac{1}{p(p-1)} \leq \sum_p\frac{2}{p^2}<\infty
  \end{eqnarray*}
  \begin{eqnarray*}
    &\implies& \lim_{s\rightarrow1^+}\log\zeta(s)=+\infty \\
    &\implies& \sum_p\frac{1}{p}=+\infty
  \end{eqnarray*}
  Hence, there are infintely many primes.
\end{proof}

\subsection{Infinite Products}

\section{Ordering of Arithmetical Functions}

\subsection{Summation Tools}

\subsection{Abel Summation}

\subsection{Euler-Maclaurin Summation}

\section{Prime Numbers}

\[ \blacktriangle \blacktriangledown \blacktriangle \]
