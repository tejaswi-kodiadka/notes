\chapter{Quadratic Curves}

\section{Conic Sections}

\begin{definition}
  A conic section, or a conic, is a curve obtained from intersection of a plane with the
  surface of a cone.
\end{definition}

Germinal Pierre Dandelin, a 19th century French-Belgian Mathematician, discovered this elegant 
proof to demonstrate that any plane that cuts through a right circular cone produces a
quadratic curve.

\begin{theorem}
  When a plane intersects a right circular cone, the curve produced will either be an ellipse, a
  parabola or a hyperbola.
\end{theorem}

\begin{proof}
  Place a sphere tangent to the intersecting plane $\pi$ and the cone such that it touches the
  plane at $F$, and the cone in a circle $C$ with centre $O$, that lies on a horizontal plane
  $\epsilon$.

  Take an aribtrary point $P$ on the curve $Q$ produced by the intersection of the plane $\pi$
  and the cone, and extend the line $VP$ from the vertex $V$ of the cone such that it meets
  the circle $C$ at point $L$. Let $D$ be the point on the intersection on the planes $\pi$
  and $\epsilon$ such that $PD$ is perpendicular to the line of intersection.

  Drop a perpendicular $PM$ on $OL$ such that $\triangle PML$ and $\triangle PMD$ are both
  right angled. Denote $\angle PLM$ as $\alpha$, and $\angle PDM$ as $\beta$.

  Consider the triangles  $\triangle PML$ and $\triangle PMD$:
  \begin{eqnarray*}
    \sin{\alpha}&=&\frac{PM}{PD}\\ 
    \quad \textrm{and} \quad \sin{\beta}&=&\frac{PM}{PL}\\
    \quad \implies \quad \frac{PL}{PD}&=&\frac{\sin{\alpha}}{\sin{\beta}}
  \end{eqnarray*}

  \begin{figure}[H]
    \center
    \tdplotsetmaincoords{70}{0}
    \begin{tikzpicture}[tdplot_main_coords]
      \begin{scope}[scale=2.5]
      \draw [line width=1.25] (2,0,0) -- (0,0,5) node[anchor=south]{$V$};
      \draw [line width=1.25] (0,0,5) -- (-2,0,0);
      \draw [line width=1.25] (2,0,0) arc (360:180:2);
      \draw [line width=1] (0,0,5) -- ({(1.2)*cos(230)},{(1.2)*sin(230)},2);
      \draw (1.5,1.2,2) -- (-2,1.2,2);
      \draw (-2,1.2,2) -- (-2.5,-2,2);
      \draw (-2.5,-2,2) -- (2,-2,2);
      \draw (2,-2,2) -- (1.5,1.2,2);
      \draw (-2,-2,2) -- (-1.5,1.2,2);
      \draw (-2,-2,2) -- (2,-2,4.5);
      \draw (-1.5,1.2,2) -- (1.3,1.2,4);
      \draw (1.3,1.2,4) -- (2,-2,4.5);
      \draw ({(-1.5)+(0.35)*(-0.5)},{(1.2)+(0.35)*(-3.2)},2)
        -- ({1.3+(0.33)*(0.7)},{1.2+(0.33)*(-3.2)},{4+(0.33)*(0.5)});
      \draw ({(-1.5)+(0.775)*(-0.5)},{(1.2)+(0.775)*(-3.2)},2)
        -- ({1.3+(0.75)*(0.7)},{1.2+(0.75)*(-3.2)},{4+(0.75)*(0.5)})
        node[anchor=south west]{$\pi$};
      \draw [dashed] (0,0,2) -- ({(1.2)*cos(230)},{(1.2)*sin(230)},2);
      \draw [dashed] ({(-1.5)+(0.35)*(-0.5)},{(1.2)+(0.35)*(-3.2)},2) -- (1.2,0,2);
      \draw [dashed] ({(0.78)*(1.2)*cos(230)},{(0.78)*(1.2)*sin(230)},{5-(0.78)*(3)})
        -- ({(0.78)*(1.2)*cos(230)},{(0.78)*(1.2)*sin(230)},2);

      \filldraw ({(1.2)*cos(230)},{(1.2)*sin(230)},2) circle (0.5pt)
        node[anchor=north east]{$L$};
      \filldraw ({(-1.5)+(0.775)*(-0.5)},{(1.2)+(0.775)*(-3.2)},2) circle (0.5pt)
        node[anchor=south east]{$D$};
      \filldraw (0,0,2) circle (0.5pt)
        node[anchor=south west]{$O$};
      \filldraw ({(0.78)*(1.2)*cos(230)},{(0.78)*(1.2)*sin(230)},{5-(0.78)*(3)})
        circle (0.5pt) node[anchor=south west]{$P$};
      \filldraw ({(0.78)*(1.2)*cos(230)},{(0.78)*(1.2)*sin(230)},2) circle (0.5pt)
        node[anchor=west]{$M$};
      \filldraw ({(-1.675)+(0.33)*(1.675+1.531)},{(0.08)+(0.33)*(-0.08+0.144)},{2+(0.33)*(-2+4.165)})
        circle (0.5pt) node[anchor=north]{$F$};

      \tdplotdrawarc[thick]{(0,0,2)}{1.2}{150}{510}{anchor=north}{$C$}
      \tdplotsetthetaplanecoords{0}
      \tdplotdrawarc[tdplot_rotated_coords,dashed, thick]
        {(0,0,{3.2+(0.4)*sqrt(10)})}{{(0.4)*sqrt(10)}}{0}{360}{}{}
      \tdplotdrawarc[tdplot_rotated_coords]
        {(2,-2,-2)}{0.2}{57}{90}{anchor=west}{$\alpha$}
      \tdplotsetrotatedcoords{0}{-35.5376777918}{0}
      \tdplotdrawarc[tdplot_rotated_coords, thick]
        {(1.55,0,2.54)}{0.96}{240}{600}{anchor=south east}{$E$}
      \tdplotsetthetaplanecoords{230}
      \tdplotdrawarc[tdplot_rotated_coords]
        {(2,1.2,0)}{0.15}{270}{335}{anchor=south west}{\small$\beta$}
      \end{scope}
    \end{tikzpicture}
    \caption{When $0<\alpha<\beta<\frac{\pi}{2}$.}
  \end{figure}

  Since $PL$ and $PF$ are both tangents from $P$ to the sphere, $PF=PL$. Therfore,
  \begin{eqnarray*}
  \frac{PF}{PD}=\frac{\sin{\alpha}}{\sin{\beta}}\\
  \implies PF=e\cdot PD
  \end{eqnarray*}
  where $e=\sin{\alpha}/\sin{\beta}$\\
  It follows from the focus - directrix definition that $Q$ will be an ellipse if
  $\alpha<\beta$, a parabola if $\alpha=\beta$, or a hyperbola if $\alpha>\beta$.
\end{proof}

\section{Group Laws on Conics}

Consider a conic section 
\[
  C=\{(x,y)\in\F^2:f(x,y)=0,f\in\F[x,y]\}
\]
where $f$ is squarefree, $deg(f)=2$, and $ch(\F)\ne 2$. Given a fixed point $O \in C$, for any
$P,Q \in C$, define a binary operation $\oplus :C \times C \rightarrow C$ by
$P \oplus Q = R$, where $R$ is such that $l_{PQ}$ is parellel to $l_{OR}$.

\begin{theorem}
  Set of points of $C$ forms a group under the binary operation $\oplus$.
\end{theorem}

\begin{proof}
  Under $\oplus$, the point $O$ serves as the identity, and when $Q$ is such that the
  line parellel to $l_{PQ}$ that passes through $O$ is tangent to the conic, that is when
  $R=O$, we get $P\oplus Q=O$. Such $Q$ is the inverse for any $P\in C$.

  To prove that $\oplus$ is associative, we'll derive expressions for $P\oplus Q$ by
  paremeterization for standard conics: for the circle $x^2+y^2=1$, for the parabloa
  $y=x^2$, and for the hyperbola $xy=1$. In the next Chapter, we will prove that all ellipses,
  hyperbolas, and parabolas are affine congruent to their respective standard forms. It will
  generalize our results to all conics.

  Let the point $P$ be $(p_1,p_2)$, $Q$ be $(q_1,q_2)$, $O$ be $(o_1,o_2)$, and $R$ be
  $(r_1,r_2)$. The slope of the line $l_{PQ}$ will be $\lambda=\frac{q_2-p_2}{q_1-p_1}$,
  assuming $P\not=Q$ (associativity would be trivial then). Let $\ell$ be the line through $O$
  with slope $\lambda$. $(r_1,r_2)$ will satisfy:
  \begin{eqnarray*}
    \lambda &=& \frac{r_2-o_2}{r_1-o_2}=\frac{q_2-p_2}{q_1-p_1} \\
    \implies r_2 &=& o_2+\mu(q_2-p_2) \text{ and} \\
    r_1 &=& o_1+\mu(q_1-p_1) 
  \end{eqnarray*}

  for some $\mu\in\mathbb{F}$.

  \begin{enumerate}[label=(\roman*)]
    \item
        \textbf{Circle}\\
        Without loss of generality, let $O=(1,0)$. Since $R$ also lies on $C$,
        $r_1^2+r_2^2=1$. i.e.

        \begin{eqnarray*}
          &&(1+\mu(q_1-p_1))^2+(0+\mu(q_2-p_2))^2=1\\
          \Longrightarrow&& \mu(\mu(q_1-p_1)^2+\mu(q_2-p_2)^2+2(q_1-p_1))=0\\
          \Longrightarrow&& \mu=0\text{ or }\mu=-\frac{2(q_1-p_1)}{(q_1-p_1)^2+(q_2-p_2)^2}
        \end{eqnarray*}

        We assume that $(q_1-p_1)^2+(q_2-p_2)^2\ne0$, leaving out the case when $P=Q$.

        \begin{figure}[H]
          \center
          \begin{tikzpicture}
            \tkzDefPoint(0,0){C}
            \tkzDefPoint(0,3.25){y}
            \tkzDefPoint(0,-3.25){y'}
            \tkzDefPoint(3.25,0){x}
            \tkzDefPoint(-3.25,0){x'}
            \tkzDefPoint(3,0){O}
            \tkzDefPoint(0,3){P}
            \tkzDefPoint(-3,0){Q}
            \tkzDefPoint(0,-3){R}
            \tkzDrawPoints[fill=black](O,P,Q,R) 
            \tkzDrawLine[<->,line width=0.5, gray](y,y')
            \tkzDrawLine[<->,line width=0.5, gray](x,x')
            \tkzDrawLine[<->, line width=1, add=0.2 and 0.2, red](P,Q)
            \tkzDrawLine[<->, line width=1, add=0.2 and 0.2, red](O,R)
            \tkzDrawCircle[color=black, line width=1](C,O)
            \tkzLabelPoints[below right](O)
            \tkzLabelPoints[above left](P)
            \tkzLabelPoints[below right](R)
            \tkzLabelPoints[above left](Q)
          \end{tikzpicture}
          \caption{$R = P \oplus Q$ when $C$ is a circle.}
        \end{figure}

        The $\mu=0$ solution corresponds to $O=R$. Considering the other solution,

        \begin{eqnarray*}
          r_1&=& 1-\frac{2(q_1-p_1)^2}{(q_1-p_1)^2+(q_2-p_2)^2}\\
             &=& \frac{(q_2-p_2)^2-(q_1-p_1)^2}{(q_1-p_1)^2+(q_2-p_2)^2}\\
             &=& \frac{1-p_1^2-q_1^2+p_1q_1-p_2q_2}{1-p_1q_1-p_2q_2}\\
             &=& \frac{(p_1q_1-p_2q_2)(1-p_1q_1-p_2q_2)}{1-p_1q_1-p_2q_2}\\
             &=& p_1q_1-p_2q_2
        \end{eqnarray*}
        and
        \begin{eqnarray*}
          r_2&=& -\frac{2(q_1-p_1)(q_2-p_2)}{(q_1-p_1)^2+(q_2-p_2)^2}\\
             &=& \frac{p_2q_2+p_2q_1-p_1p_2-q_1q_2}{1-p_1q_1-p_2q_2}\\
             &=& \frac{(p_1q_2+p_2q_1)(1-p_1q_1-p_2q_2)}{1-p_1q_1-p_2q_2}\\
             &=& p_1q_2+p_2q_1
        \end{eqnarray*}

        $$\Longrightarrow P\oplus Q=(p_1q_1-p_2q_2,p_1q_2+p_2q_1)$$
    \item
        \textbf{Parabola}\\
        Without loss of generality, let $O=(0,0)$. The points of the standard parabloa can
        be parameterized as $(t,t^2)$. Let $P=(p,p^2)$, $Q=(q,q^2)$, and $R=(r,r^2)$.
        Substituting them in $\lambda$:

        \begin{eqnarray*}
          \lambda&=& \frac{r^2}{r}=\frac{q^2-p^2}{q-p}\\
          &\Longrightarrow& r=p+q \\
          &\Longrightarrow& P\oplus Q=(p+q,(p+q)^2)
        \end{eqnarray*}
 
        \begin{figure}[H]
          \center
          \begin{tikzpicture}
            \draw[domain=-2.25:2.25, line width=1] plot (\x, {(\x)^2});
            \tkzDefPoint(0,0){C}
            \tkzDefPoint(0,4.75){y}
            \tkzDefPoint(0,-0.5){y'}
            \tkzDefPoint(2,0){x}
            \tkzDefPoint(-2,0){x'}
            \tkzDefPoint(0,0){O}
            \tkzDefPoint(-2,4){P}
            \tkzDefPoint(1,1){Q}
            \tkzDefPoint(-1,1){R}
            \tkzDrawPoints[fill=white](O,P,Q,R) 
            \tkzDrawLine[<->,line width=0.5, gray](y,y')
            \tkzDrawLine[<->,line width=0.5, gray](x,x')
            \tkzDrawLine[<->, line width=1, add=0.2 and 0.2, red](P,Q)
            \tkzDrawLine[<->, line width=1, add=0.5 and 0.5, red](O,R)
            \tkzLabelPoints[below left](O)
            \tkzLabelPoints[below left](P)
            \tkzLabelPoints[above right](Q)
            \tkzLabelPoints[below left](R)
          \end{tikzpicture}
          \caption{$R = P \oplus Q$ when $C$ is the standard parabola.}
        \end{figure}
    \item
        \textbf{Hyperbola}\\
        Without loss of generality, let $O=(1,1)$.The points of the standard hyperbola can
        be parameterized as $(t,\frac{1}{t})$. Let $P=(p,\frac{w}{p})$, $Q=(q,\frac{1}{q})$,
        and $R=(r,\frac{1}{r})$. Substituting them in $\lambda$:

        \begin{eqnarray*}
          \lambda&=& \frac{\frac{1}{r}-1}{r-1}=\frac{\frac{1}{q}-\frac{1}{p}}{p-q}\\
          &\Longrightarrow& r=pq\\
          &\Longrightarrow& P\oplus Q=(pq,\frac{1}{pq})
        \end{eqnarray*}

        \begin{figure}[H]
          \center
          \begin{tikzpicture}
            \draw[domain=0.3:5, line width=1] plot (\x, 1/\x);
            \draw[domain=-5:-0.3, line width=1] plot (\x, 1/\x);
            \tkzDefPoint(0,0){C}
            \tkzDefPoint(0,2.5){y}
            \tkzDefPoint(0,-2.5){y'}
            \tkzDefPoint(4,0){x}
            \tkzDefPoint(-4,0){x'}
            \tkzDefPoint(1,1){O}
            \tkzDefPoint(2,0.5){P}
            \tkzDefPoint(-2,-0.5){Q}
            \tkzDefPoint(-4,-0.25){R}
            \tkzDrawPoints[fill=black](O,P,Q,R)
            \tkzDrawLine[<->,line width=0.5, gray](y,y')
            \tkzDrawLine[<->,line width=0.5, gray](x,x')
            \tkzDrawLine[<->, line width=1, add=0.4 and 0.6, red](P,Q)
            \tkzDrawLine[<->, line width=1, add=0.4 and 0.2, red](O,R)
            \tkzLabelPoints[above right](O)
            \tkzLabelPoints[above](P)
            \tkzLabelPoints[below](Q)
            \tkzLabelPoints[below](R)
          \end{tikzpicture}
          \caption{$R = P \oplus Q$ when $C$ is the standard hyperbola.}
        \end{figure}
  \end{enumerate}
  From the expressions obtained for $P\oplus Q$, in all three cases it can be easily proved
  that $P\oplus(Q\oplus R)=(P\oplus Q)\oplus R$, $\forall P,Q,R \in C$.
\end{proof}

\begin{remark}
  It can also be shown that the group $\langle C,\oplus\rangle$ is isomorphic to some other
  well known groups in each case:

  \begin{itemize}
  \item When $C$ is an ellipse, $\langle C,\oplus\rangle\cong\langle S^1,\cdot\rangle$ \\
    where $S^1=\{e^{i\theta}\in\C:\theta\in[0,2\pi)\}$.
  \item When $C$ is a parabola, $\langle C,\oplus\rangle\cong\langle\R,+\rangle$.
  \item When $C$ is a hyperbola, $\langle C,\oplus\rangle\cong\langle\R^{\times},\cdot\rangle$.
  \end{itemize}
\end{remark}

\section{Solutions for Diophantine Equations}

Consider the conic $C=\{(x,y)\in\Q : x^2+y^2=1\}$, and $P=(1,0)\in C$. Let $l_m$ be the line
with slope $m\in Q$, passing through $P$ and another point $Q=(x,y)\in C$. The coorrdinates
of $Q$ can be found by substituting $y=m(x-1)$.
$$x^2+m^2(x-1)^2-1=(1+m^2)x^2-2m^2x-(1-m^2)=0$$
using the quadratic formula,
$$x=\frac{m^2\pm1}{1+m^2}$$
from the non-trivial solution, we get $x=\frac{m^2-1}{m^2+1}$ and $y=\frac{-2m}{m^2+1}$.
substiting these values in the equation for the conic,

\begin{eqnarray*}
  (\frac{m^2-1}{m^2+1})^2+(\frac{-2m}{m^2+1})^2=1\\
  \Longrightarrow (m^2-1)^2+(2m)^2=(m^2+1)^2
\end{eqnarray*}

This equation will produce integer solutions for $x^2+y^2=1$, though not all of them. Rational
or integer solutions for any equations of the form $ax^2+by^2=c$, where $a,b,c\in Q$ can be
similarly produced.
\[ \blacktriangle \blacktriangledown \blacktriangle \]
