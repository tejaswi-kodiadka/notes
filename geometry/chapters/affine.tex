\chapter{Affine Geometry}

\section{Affine Space}

\begin{definition}
  A set $\mathcal{E}$ is endowed with the structure of an affine space by a vector space $E$
  and a mapping $\Theta$ that associates a vector of $E$ with any ordered pair of points in
  $\mathcal{E}$,

  \begin{eqnarray*}
    \mathcal{E} \times \mathcal{E} &\longrightarrow & E \\
    (A,B) &\longmapsto & \overrightarrow{AB}
  \end{eqnarray*}

  such that:

  \begin{itemize}
    \item[-]for any point $A$ of $\mathcal{E}$, the partial map $\Theta_A : B \mapsto
        \overrightarrow{AB}$ is a bijection from $\mathcal{E}$ to $E$;
    \item[-]for any points $A$, $B$, and $C$ in $\mathcal{E}$, we have $\overrightarrow{AB}=
        \overrightarrow{AC}+\overrightarrow{CB}$.
  \end{itemize}

  The vector space $E$ is called the direction of $\mathcal{E}$, or its underlying vector
  space. The elements of $\mathcal{E}$ are called points, and the dimension of $\mathcal{E}$
  is defined to be the dimension of the vector space $E$.
\end{definition}

A subset $\mathcal{F}$ of $\mathcal{E}$ is an affine subspace if it is empty, or if it
contains a point $A$ such that $\Theta_A(\mathcal{F})$ is a vector subspace of $E$.

\subsection{Affine Transfromations}

\begin{definition}
  Let $\mathcal{E}$ and $\mathcal{F}$ be two affine spaces directed respectively by the vector
  spaces $E$ and $F$. A transformation $\varphi :\mathcal{E}\rightarrow\mathcal{F}$ is said to be
  affine if there exists a point $O\in\mathcal{E}$ and a linear transformation $f:E\rightarrow F$
  such that $\forall M\in E$ $f(\overrightarrow{OM})=\overrightarrow{\varphi(O)\varphi(M)}$.
\end{definition}

When $E=F=\R^2$, it can be proved that the map $f$ is of the form $f(x)=Ax+b$, where
$b\in\R^2$, and $A$ is a $2\times2$ invertible matrix. The set of all affine transformations
of $\R^2$ is denoted by $A(2)$. If $A$ is orthagonal, the map $f$ would produce a Euclidean
transformation, where all distances and angles are preserved. From the definition, all affine
transformations:
\begin{enumerate}
  \item Map straight lines into straight lines;
  \item Map parellel straight lines into parellel straight lines;
  \item Preserve ratios of lengths along a given straight line.
\end{enumerate}
Translations and Rotations are examples of affine transformations in 2 dimensional affine
spaces.

\begin{remark}
  For the affine transformations $t:\R^2\rightarrow\R^2$, $t(x)=Ax+b$, the inverse is given by
  $t^{-1}(x)=A^{-1}x-A^{-1}b$
\end{remark}

\section{Fundamental Theorem of Affine Geometry}

\begin{lemma}
  The points $(0,0)$, $(0,1)$, and $(1,0)$ can be mapped into any three non-collinear points
  $p$, $q$, and $r$ by a unique affine transformation.
\end{lemma}

\begin{proof}
  Any affine transformation
  $t:\R^2\rightarrow\R^2$ has the form $t(x)=Ax+b$. Where
  \[
    A=
    \left(\begin{array}{cc}
      a & b \\
      c & d \\
     \end{array}\right)
     \text{ and }
     b=
     \left(\begin{array}{c}
         e \\
         f \\
      \end{array}\right)
  \]
  The images of the points $(1,0)$ and $(0,1)$ are $(a+e,c+f)$ and $(b+e, d+f)$ respectively.
  For any $p$, $q$, and $r$, we can choose $t$ to be such that $(e,f)=p$, $(a,e)=q-p$, and
  $(b,d)=r-p$. For some $t'$ such that $t'((0,0))=p$, $t'((1,0))=q$, and
  $t'((0,1))=r$, we would have $t(x)-t'(x)=0$ $\forall x$. Hence, such transformation $t$ is
  unique.
\end{proof}

Using the above lemma, we prove the two-dimensional analog of the so called Fundamental
theorem of Affine Geometry.

\begin{theorem}[Fundamental theorem of Affine Geometry]
  For any two sets of three non-collinear points, $\{p,q,r\}$ and $\{p',q',r'\}$ in $\R^2$,
  there exists a unique affine transformation $t$ which maps $p$, $q$, and $r$ to $p'$, $q'$,
  and $r'$, respectively.
\end{theorem}

\begin{definition}
  Two subsets of an affine space are said to be affine congruent if there exists an affine
  transformation which takes one to the other.
\end{definition}

\begin{proof}
  Let $t_1$ be the affine transformation which maps $(0,0)$, $(0,1)$, and $(1,0)$ to the
  points $p$, $q$, and $r$ respectively, and let $t_2$ be the affine transformation which maps
  $(0,0)$, $(0,1)$, and $(1,0)$ to the points $p'$, $q'$, and $r'$ respectively. Then, the map
  $t=t_2\circ t_1^{-1}$ is an affine transformation, and it maps $p$, $q$, and $r$ to $p'$,
  $q'$, and $r'$ respectively.

  Suppose $t$ and $s$ are both affine transformations which map $p$, $q$, and $r$ to $p'$,
  $q'$ and $r'$ respectively. Then the composites $t\circ t_1$, and $s\circ t_1$ both map the
  points $(0,0)$, $(0,1)$, and $(1,0)$ to the points $p'$, $q'$, and $r'$ respectively.
  Since such map is unique, $t\circ t_1$=$s\circ t_1$. Composing both of them with $t_1^{-1}$
  to the right, it follows that $t$=$s$.
\end{proof}

\begin{corollary}
  All triangles are affine congruent.
\end{corollary}

\section{Affine Congruence of Conics}

\begin{lemma}
  Affine transformations map ellipses to ellipses, parabolas to parabolas, and hyperbolas to
  hyperbolas.
\end{lemma}

\begin{proof}
  Consider the non-degenerate conic with the equation
  \[
    Ax^2+Bxy+Cy^2+Fx+Gy+H=0,
  \]
  and its image under the transformation $t$ given by $t(m)=Jm+k$, for some invertible matrix
  $J$, and $k\in\R^2$. The inverse transformation given by $t^{-1}(n)=J^{-1}n-J^{-1}k$ can be
  written in the form
  \[
    \left(\begin{array}{c}
        x \\
        y \\
    \end{array}\right)
    =
    \left(\begin{array}{cc}
        p & q \\
        r & s \\
    \end{array}\right)
    \left(\begin{array}{c}
        x' \\
        y' \\
    \end{array}\right)
    +
    \left(\begin{array}{c}
        u \\
        v \\
    \end{array}\right)
  \]
  where $m=(x,y)$, and $n=(x',y')$. It follows that $x=px'+qy'+u$, and $y=rx'+sy'+v$. If these
  expressions are substituted into the equaiton of the conic, we get back a second degree
  equation. So the image of a conic must be a conic. It cannot be degenerate since affine
  transformations map lines into lines, and if the resultant conic is a line, then the inverse
  would map it into a degenrate conic. But this cannot happen since our original conic is
  non-degenerate.

  If we substitute $x$ and $y$ to the equation of the original conic, it turns out that the
  discriminant obtained is just
  \[
    (ps-rq)^2(B^2-4AC)
  \]
  Where $B^2-4AC$ is the discriminant of the original conic. \\
  Since $(ps-rq)^2>0$, the sign of the discriminant does not change. Hence the type of conic
  is also unchanged.
\end{proof}

\begin{theorem}
  All ellipses are congruent to each other.
\end{theorem}

\begin{proof}
  Consider a general elliipse centered at the origin, with aligned axes, $E_1:$
  \[\frac{x^2}{a^2}+\frac{y^2}{b^2}=1 \]
  With a translation to move the center to the origin and a rotation to align the axes, any
  ellipse can be transformed into this form.
  If we apply the affine transformation $t_1:(x,y)\mapsto (x',y')$, where
  \[
    \left(\begin{array}{c}
        x' \\
        y' \\
    \end{array}\right)
    =
    \left(\begin{array}{cc}
        \frac{1}{a} & 0           \\
        0           & \frac{1}{b} \\
    \end{array}\right)
    \left(\begin{array}{c}
        x \\
        y \\
    \end{array}\right)
  \]
  Then the equation becomes $(x')^2+(y')^2=1$. The map $t=t_2\circ t_1^{-1}$ takes the ellipse
  $E_1$ to $E_2$.
\end{proof}

\begin{theorem}
  All hyperbolas are congruent to each other.
\end{theorem}

\begin{proof}
  Consider a general hyperbola centerd at the origin with aligned axes, $H_1:$
  \[\frac{x^2}{a^2}-\frac{y^2}{b^2}=1 \]
  With a translation to move the center to the origin and a rotation to align the axes, any
  hyperbola can be transformed into this form.
  If we apply the affine transformation $t_1:(x,y)\mapsto (x',y')$, where
  \[
    \left(\begin{array}{c}
        x' \\
        y' \\
    \end{array}\right)
    =
    \left(\begin{array}{cc}
        \frac{1}{a} & -\frac{1}{a} \\
        \frac{1}{b} & \frac{1}{b}  \\
    \end{array}\right)
    \left(\begin{array}{c}
        x \\
        y \\
    \end{array}\right)
  \]
  Then the equation becomes $x'y'=1$. The map $t=t_2\circ t_1^{-1}$ takes the hyperbola
  $H_1$ to $H_2$.
\end{proof}

\begin{theorem}
  All parabolas are congruent to each other.
\end{theorem}

\begin{proof}
  Consider a general parabola centered at the origin with aligned axes, $P_1:y^2=ax$. With a
  translation to move the center to the origin and a rotation to align the axes, any
  parabola can be transformed into this form.
  If we apply the affine transformation $t_1:(x,y)\mapsto (x',y')$, where
  \[
    \left(\begin{array}{c}
        x' \\
        y' \\
    \end{array}\right)
    =
    \left(\begin{array}{cc}
        \frac{1}{a} & 0            \\
        0           & \frac{1}{a}  \\
    \end{array}\right)
    \left(\begin{array}{c}
        x \\
        y \\
    \end{array}\right)
  \]
  Then the equation becomes $(y')^2=x'$. The map $t=t_2\circ t_1^{-1}$ takes the parabola
  $P_1$ to $P_2$.
\end{proof}
\[ \blacktriangle \blacktriangledown \blacktriangle \]
